% Definição do tipo de documento e configurações básicas
\documentclass[a4paper,12pt]{report}

% Pacotes necessários para o documento
% inputenc: Suporte a caracteres UTF-8
% fontenc: Codificação de fonte para caracteres especiais
% babel: Suporte para Português
% graphicx: Suporte para imagens
% listings: Formatação de código fonte
% color: Suporte a cores
% hyperref: Links clicáveis no documento
\usepackage[utf8]{inputenc}
\usepackage[T1]{fontenc}
\usepackage[portuguese]{babel}
\usepackage{graphicx}
\usepackage{listings}
\usepackage{color}
\usepackage{hyperref}
\usepackage{float}

% Configuração do hyperref para hiperligações coloridas
\hypersetup{
    colorlinks=true,
    linkcolor=blue,
    filecolor=magenta,
    urlcolor=blue,
    citecolor=red
}

% Configuração do listings para apresentação de código fonte
% Definições para formatação e suporte a caracteres especiais portugueses
\lstset{
    basicstyle=\ttfamily\small,
    numbers=left,
    numberstyle=\tiny,
    numbersep=5pt,
    breaklines=true,
    frame=single,
    keywordstyle=\color{blue},
    commentstyle=\color{green},
    stringstyle=\color{red},
    tabsize=2,
    inputencoding=utf8,
    extendedchars=true,
    escapeinside={(*@}{@*)}
   ,literate={á}{{\'a}}1 {à}{{\`a}}1 {ã}{{\~a}}1 {â}{{\^a}}1
             {é}{{\'e}}1 {ê}{{\^e}}1
             {í}{{\'i}}1 {ì}{{\`i}}1
             {ó}{{\'o}}1 {õ}{{\~o}}1 {ô}{{\^o}}1
             {ú}{{\'u}}1 {ü}{{\"u}}1
             {ç}{{\c{c}}}1
             {_}{{\textunderscore}}1
}

\begin{document}

% Página de título do documento
\title{Sistema de Gestão do Espaço Social\\
       \large Relatório Técnico}
\author{Pedro Vilas Boas - 25453 \\ Ricardo Marques - 25447}
\date{\today}

\maketitle

% Resumo do trabalho
\begin{abstract}
Este trabalho apresenta o desenvolvimento de um sistema de gestão do espaço social, implementado em linguagem C. O sistema foi desenvolvido como parte da unidade curricular de Laboratórios de Informática, com foco em boas práticas de programação e estruturas de dados eficientes. O projeto utiliza estruturas de dados dinâmicas (listas ligadas) para gerir informações sobre funcionários, ementas e escolhas de refeições. São apresentadas as principais funcionalidades implementadas, incluindo gestão de funcionários, controlo de refeições e cálculo de médias de calorias, bem como os algoritmos e estruturas de dados utilizados em cada componente do sistema.
\end{abstract}

% Índices automáticos
\tableofcontents
\listoffigures
\listoftables

% Capítulo 1: Introdução ao projeto
\chapter{Introdução}

% Secção que detalha os objetivos do projeto
\section{Objectivos}
Este trabalho representa a continuação e evolução do projeto desenvolvido na Unidade Curricular de Programação Imperativa, agora no contexto da disciplina de Laboratórios Informáticos. O objectivo principal é melhorar e expandir o sistema anteriormente desenvolvido, aplicando conceitos avançados de desenvolvimento de programação e boas práticas de engenharia.

% Secção que apresenta uma visão geral do sistema
\section{Descrição do Sistema}
O sistema de gestão do espaço social foi desenvolvido para satisfazer as necessidades de controlo e organização de refeições num ambiente empresarial. O projecto foi implementado em linguagem C, utilizando conceitos avançados de programação como alocação dinâmica de memória, estruturas de dados ligadas e manipulação de ficheiros.

% Subsecção que lista as principais funcionalidades
\subsection{Funcionalidades Principais}
\begin{enumerate}
    \item \textbf{Gestão de Funcionários}
    \begin{itemize}
        \item Carregar dados de funcionários
        \item Listar informações dos funcionários
    \end{itemize}

    \item \textbf{Controlo de Ementas}
    \begin{itemize}
        \item Carregar dados de ementas
        \item Listar informações de ementas
    \end{itemize}

    \item \textbf{Gestão de Escolhas}
    \begin{itemize}
        \item Carregar dados de escolhas dos funcionários
        \item Listar refeições por dia
        \item Listar utentes ordenados
        \item Consultar refeições de um utente
        \item Calcular médias de calorias
    \end{itemize}
\end{enumerate}

% Capítulo 2: Detalhes do desenvolvimento
\chapter{Desenvolvimento}

% Secção que contextualiza o problema a resolver
\section{Descrição do Problema}
O problema a ser resolvido envolve a criação de um sistema de gestão do espaço social. O sistema deve permitir a entrada de dados através de ficheiros de texto, oferecendo funcionalidades como listagem de refeições, consulta de refeições por utente, e cálculo de médias de calorias.

% Secção que detalha as estruturas de dados utilizadas
\section{Estruturas de Dados}
O sistema utiliza estruturas de dados dinâmicas para gerir as informações de forma eficiente. As principais estruturas implementadas são:

% Subsecção específica para a estrutura de funcionários
\subsection{Estrutura de Funcionários}
% Definição da estrutura básica de funcionários
\begin{lstlisting}[caption={Estrutura de dados para Funcionários},label=lst:funcionario]
typedef struct {
    int numero;        // Número único do funcionário
    char nome[50];     // Nome completo
    int nif;          // Número de Identificação Fiscal
    int telefone;     // Número de telefone
} Funcionario;
\end{lstlisting}
% Definição do nó para lista ligada de funcionários
\begin{lstlisting}[caption={Estrutura do Nó da Lista de Funcionários},label=lst:noFuncionario]
typedef struct NodeFunc {
    Funcionario func;           // Dados do funcionário
    struct NodeFunc* prox;      // Ponteiro para o próximo nó
} NodeFunc;
\end{lstlisting}

% Subsecção específica para a estrutura de ementas
\subsection{Estrutura de Ementas}
% Definição da estrutura de ementas com pratos e calorias
\begin{lstlisting}[caption={Estrutura de dados para Ementas},label=lst:ementa]
typedef struct {
    char dia[10];   // Dia da semana (ex: 2feira)
    char data[10]; // Data no formato DD.MM.AAAA
    struct {
        char nome[50]; // Nome do prato de carne
        int calorias;  // Calorias do prato de carne
    } prato_carne;
    struct {
        char nome[50]; // Nome do prato de peixe
        int calorias;  // Calorias do prato de peixe
    } prato_peixe;
} Ementa;
\end{lstlisting}
% Definição do nó para lista ligada de ementas
\begin{lstlisting}[caption={Estrutura do Nó da Lista de Ementas},label=lst:noEmenta]
typedef struct NodeEmenta {
    Ementa ementa;
    struct NodeEmenta* prox;
} NodeEmenta;
\end{lstlisting}

% Subsecção específica para a estrutura de escolhas
\subsection{Estrutura de Escolhas}
% Definição da estrutura de escolhas dos utentes
\begin{lstlisting}[caption={Estrutura de dados para Escolhas},label=lst:escolha]
typedef struct {
    char dia[10];         // Dia da semana (ex: 2feira)
    int num_funcionario;    // Número do funcionário
    char tipo_prato;     // 'C' para carne, 'P' para peixe
} Escolha;
\end{lstlisting}
% Definição do nó para lista ligada de escolhas
\begin{lstlisting}[caption={Estrutura do Nó da Lista de Escolhas},label=lst:noEscolha]
typedef struct NodeEscolha {
    Escolha escolha;
    struct NodeEscolha* prox;
} NodeEscolha;
\end{lstlisting}

% Capítulo 3: Detalhes da implementação
\chapter{Implementação}

% Secção que explica a organização do código fonte
\section{Organização do Código}
O código fonte está organizado em vários ficheiros:
\begin{itemize}
    \item \texttt{estruturas.h}: Definição das estruturas de dados principais
    \item \texttt{funcoes.c}: Implementação das funções principais
    \item \texttt{main.c}: Programa principal e interface com utilizador
\end{itemize}

% Subsecção que detalha as estruturas de dados implementadas
\subsection{Estruturas de Dados}
As principais estruturas implementadas são:
\begin{itemize}
    \item \textbf{Funcionario}: Armazena dados dos funcionários (número, nome, NIF, telefone).
    \item \textbf{Ementa}: Mantém informações das refeições (dia da semana, data, nome e calorias dos pratos).
    \item \textbf{Escolha}: Regista escolhas de refeições (dia da semana, número do funcionário, tipo de prato ('C' para carne, 'P' para peixe)).
\end{itemize}

% Subsecção que descreve as principais funções do sistema
\subsection{Funções Principais}
\begin{itemize}
    \item \textbf{Gestão de Funcionários}:
    \begin{itemize}
       \item \texttt{carregarFuncionarios}: Carrega os dados dos funcionários do ficheiro (O(n)). Esta função lê o ficheiro especificado, processa cada linha para extrair os dados do funcionário e cria um novo nó na lista encadeada de funcionários. Retorna o primeiro nó da lista.
        \item \texttt{criarNoFunc}: Cria um novo nó da lista de funcionários (O(1)). Esta função aloca dinamicamente a memória para um novo nó da lista de funcionários e inicializa seus campos com os dados do funcionário passados como parâmetro.
    \end{itemize}
    \item \textbf{Gestão de Ementas}:
    \begin{itemize}
        \item \texttt{carregarEmentas}: Carrega os dados das ementas do ficheiro (O(n)). Esta função lê o ficheiro especificado, processa cada linha para extrair os dados da ementa e cria um novo nó na lista encadeada de ementas. Retorna o primeiro nó da lista.
    \end{itemize}
    \item \textbf{Gestão de Escolhas}:
    \begin{itemize}
        \item \texttt{carregarEscolhas}: Carrega os dados das escolhas do ficheiro (O(n)). Esta função lê o ficheiro especificado, processa cada linha para extrair os dados de escolha do funcionário e cria um novo nó na lista encadeada de escolhas. Retorna o primeiro nó da lista.
      \end{itemize}
\end{itemize}

% Subsecção sobre gestão de memória
\subsection{Gestão de Memória}
\begin{itemize}
    \item Alocação dinâmica para todas as estruturas usando \texttt{malloc}.
    \item Libertação adequada de memória ao finalizar o programa através da função \texttt{liberarMemoria}.
    \item A verificação de erros em operações de memória é realizada através da comparação do resultado da função \texttt{fopen} com \texttt{NULL}.
\end{itemize}

% Subsecção sobre validações
\subsection{Validações}
\begin{itemize}
    \item O programa valida se os arquivos foram abertos com sucesso através da função \texttt{fopen}, retornando NULL caso a abertura do ficheiro falhe.
    \item A escolha das opções do menu é validada através da comparação da entrada do utilizador com os casos definidos no \texttt{switch} da função \texttt{main}.
    \item  No carregamento de dados dos ficheiros, nenhuma validação dos dados é realizada, apenas é feita a leitura e o seu armazenamento nas estruturas.
     \item Os dados introduzidos pelo utilizador (dia da semana, data inicial e final) são lidos através da função \texttt{scanf}.
\end{itemize}

% Tabela que resume a organização dos ficheiros do projeto
\begin{figure}[htb]
    \centering
    \begin{tabular}{|l|p{10cm}|}
        \hline
        \textbf{Ficheiro} & \textbf{Responsabilidade} \\
        \hline
        main.c & Interface com utilizador, menu principal e controlo de fluxo do programa \\
        \hline
         estruturas.h & Definição das estruturas de dados e tipos utilizados, além de protótipos das funções \\
        \hline
        funcoes.c & Implementação das operações principais do sistema \\
        \hline
    \end{tabular}
    \caption{Organização dos ficheiros do projeto}
    \label{fig:ficheiros}
\end{figure}

% Subsecção que descreve os principais algoritmos utilizados
\subsection{Principais Algoritmos}
As principais operações são baseadas em percursos nas listas ligadas.

% Subsubsecção que detalha o algoritmo de carregamento de dados
\subsubsection{Carregamento de Dados}
O sistema implementa um algoritmo que percorre o ficheiro de dados linha a linha.

% Exemplo de código para o algoritmo de carregamento de dados
\begin{lstlisting}[caption={Algoritmo de carregamento de dados},label=lst:carregar]
NodeFunc* carregarFuncionarios(const char* filename) {
    NodeFunc* head = NULL;
    NodeFunc* atual = NULL;
    FILE* file = fopen(filename, "r");
    char linha[256];
    
    if (file == NULL) {
        printf("Erro ao abrir ficheiro %s\n", filename);
        return NULL;
    }
    
    while (fgets(linha, sizeof(linha), file)) {
        Funcionario func;
        // Parse da linha
        // Criar novo nó e inserir na lista
        if(parseLinha(linha, &func)){
            NodeFunc* novoNo = criarNoFunc(func);
            if (!head) {
                head = novoNo;
                atual = novoNo;
            } else {
                atual->prox = novoNo;
                atual = novoNo;
            }
        }
    }
    
    fclose(file);
    return head;
}
\end{lstlisting}
Este algoritmo, implementado na função \texttt{carregarFuncionarios}, inicia com a abertura do ficheiro especificado. Caso o ficheiro não possa ser aberto, retorna NULL. Em seguida, percorre linha a linha, usando a função \texttt{fgets}. Cada linha é lida, e um novo nó é criado e inserido na lista encadeada. A função retorna o primeiro nó da lista.

% Subsubsecção que detalha o algoritmo de cálculo de estatísticas
\subsubsection{Cálculo de Estatísticas}
O cálculo de médias é feito através de uma interação pelas listas de ementas e escolhas, de forma a que seja calculada a média de calorias por dia.

% Exemplo de código para o algoritmo de cálculo de estatísticas
\begin{lstlisting}[caption={Algoritmo de cálculo de estatísticas},label=lst:medias]
void calcularMediasCalorias(NodeEmenta* ementas, NodeEscolha* escolhas,
                           const char* data_inicio, const char* data_fim) {
    // Para cada dia
    NodeEmenta* ementa = ementas;
    while (ementa) {
        int total_calorias = 0;
        int num_refeicoes = 0;
        NodeEscolha* escolha = escolhas;
        while (escolha) {
            if (strcmp(escolha->escolha.dia, ementa->ementa.dia) == 0) {
                if (escolha->escolha.tipo_prato == 'C')
                    total_calorias += ementa->ementa.prato_carne.calorias;
                else
                    total_calorias += ementa->ementa.prato_peixe.calorias;
                num_refeicoes++;
            }
            escolha = escolha->prox;
        }
        //Calcula e exibe a média
        if (num_refeicoes > 0) {
            float media = (float)total_calorias / num_refeicoes;
            printf("%-6s | %.2f\n", ementa->ementa.dia, media);
        }
        ementa = ementa->prox;
    }
}
\end{lstlisting}
Este algoritmo, implementado na função \texttt{calcularMediasCalorias}, itera pela lista de ementas, e para cada dia, percorre a lista de escolhas para calcular o total de calorias consumidas, utilizando os dados de calorias dos pratos da ementa. Em seguida, calcula a média de calorias para aquele dia, apresentando o resultado no output.

% Capítulo 4: Exemplos de uso do sistema
\chapter{Exemplos de Uso do Sistema}

% Secção que apresenta exemplos de uso básicos do sistema
\section{Operações Básicas}

% Subsecção que detalha o registo de um funcionário
\subsection{Registo de Funcionário}
Exemplo de como adicionar um novo funcionário ao sistema:

% Exemplo de código para o registo de um funcionário
\begin{lstlisting}[caption={Exemplo de registo de funcionário},label=lst:exemplo_registo_func]
// Dados do novo funcionário (este código é apenas exemplificativo, as informações são carregadas de ficheiros)
int numero = 1001;
char nome[] = "João Silva";
int nif = 123456789;
int telefone = 912345678;

// O novo nó é criado e adicionado à lista através da função carregarFuncionarios
\end{lstlisting}
Este excerto de código demonstra como os dados de um funcionário (número, nome, NIF e telefone) são estruturados e carregados através do ficheiro \texttt{pessoas.txt} para a lista de funcionários, utilizando a função \texttt{carregarFuncionarios}.

% Subsecção que detalha o registo de uma ementa
\subsection{Registo de Ementa}
Exemplo de como registar uma nova ementa:

% Exemplo de código para o registo de uma ementa
\begin{lstlisting}[caption={Exemplo de registo de ementa},label=lst:exemplo_ementa_reg]
// Dados da nova ementa (este código é apenas exemplificativo, as informações são carregadas de ficheiros)
char dia[] = "2feira";
char data[] = "05.01.2025";
char prato_carne[] = "Bacalhau à Brás";
int calorias_carne = 850;
char prato_peixe[] = "Salmão Grelhado";
int calorias_peixe = 700;
// O novo nó é criado e adicionado à lista através da função carregarEmentas
\end{lstlisting}
Este exemplo ilustra como os dados de uma ementa (dia da semana, data e nome e calorias do prato de carne e peixe) são estruturados e carregados através do ficheiro \texttt{ementas.txt} para a lista de ementas, utilizando a função \texttt{carregarEmentas}.

% Subsecção que detalha o registo de uma escolha
\subsection{Registo de Escolha}
Exemplo de como registar a escolha de um funcionário:

% Exemplo de código para o registo de uma escolha
\begin{lstlisting}[caption={Exemplo de registo de escolha},label=lst:exemplo_escolha_reg]
// Dados da escolha (este código é apenas exemplificativo, as informações são carregadas de ficheiros)
char dia[] = "2feira";
int num_func = 1001;
char tipo_prato = 'C'; // 'C' para carne e 'P' para peixe

// A nova escolha é adicionada à lista através da função carregarEscolhas
\end{lstlisting}
Este excerto de código mostra como os dados de escolha do funcionário (dia da semana, número do funcionário e tipo de prato) são estruturados e carregados através do ficheiro \texttt{menu\textunderscore escolhido.txt} para a lista de escolhas, utilizando a função \texttt{carregarEscolhas}.

% Secção que apresenta exemplos de uso completos do sistema
\section{Casos de Uso Completos}

% Subsecção que detalha a gestão diária de refeições
\subsection{Gestão Diária de Refeições}
Exemplo de como o sistema pode ser usado para listar as refeições de um dia:

% Exemplo de código para a gestão diária de refeições
\begin{lstlisting}[caption={Exemplo de gestão diária de refeições},label=lst:exemplo_gestao_ref]
// Dados do dia (este código é apenas exemplificativo, as informações são carregadas de ficheiros)
char dia[] = "2feira";

// A lista de refeições é carregada através da função carregarRefeicoes
\end{lstlisting}
Este exemplo ilustra como o sistema pode ser usado para listar as refeições de um dia específico, carregando a lista de refeições através da função \texttt{carregarRefeicoes}.

% Capítulo 5: Conclusão
\chapter{Conclusão}

% Secção que resume as principais contribuições do projeto
\section{Contribuições}
O desenvolvimento deste sistema trouxe diversas contribuições importantes:
\begin{itemize}
    \item Implementação de um sistema completo de gestão de refeições.
    \item Utilização de estruturas de dados dinâmicas para gerir informações.
    \item Desenvolvimento de algoritmos para processamento de dados.
    \item Aplicação prática de conceitos de programação em C.
\end{itemize}

% Secção que discute os desafios encontrados durante o desenvolvimento
\section{Desafios Encontrados}
Durante o desenvolvimento, encontrámos vários desafios:
\begin{itemize}
    \item Gestão de memória em estruturas dinâmicas.
    \item Validação e tratamento de dados de entrada.
    \item Implementação de algoritmos.
    \item Manipulação de ficheiros.
\end{itemize}

% Secção que destaca o valor acrescentado do projeto
\section{Valor Acrescentado}
O projeto proporcionou várias melhorias em relação à versão anterior:
\begin{itemize}
    \item Melhor organização do código.
    \item Interface mais intuitiva.
    \item Maior eficiência no processamento de dados.
    \item Melhor gestão de memória.
\end{itemize}

% Secção que sugere trabalhos futuros
\section{Trabalho Futuro}
Para futuras versões do sistema, sugerimos:
\begin{itemize}
    \item Implementação de uma interface gráfica.
    \item Adição de funcionalidades de exportação de relatórios.
    \item Otimização adicional dos algoritmos de pesquisa.
    \item Implementação de um sistema de cópias de segurança automáticas.
\end{itemize}

\end{document}